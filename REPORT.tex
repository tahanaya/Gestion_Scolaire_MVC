%==============================================================================
% RAPPORT DE PROJET - APPLICATION DE GESTION SCOLAIRE
% Niveau : Ingénieur
%==============================================================================
\documentclass[a4paper,12pt,twoside]{report}

%------------------------------------------------------------------------------
% PACKAGES
%------------------------------------------------------------------------------
\usepackage[utf8]{inputenc}
\usepackage[T1]{fontenc}
\usepackage[french]{babel}
\usepackage{graphicx}
\usepackage{float}
\usepackage{listings}
\usepackage{xcolor}
\usepackage{hyperref}
\usepackage{geometry}
\usepackage{fancyhdr}
\usepackage{titlesec}
\usepackage{enumitem}
\usepackage{array}
\usepackage{tabularx}
\usepackage{booktabs}
\usepackage{amsmath}
\usepackage{caption}
\usepackage{subcaption}
\usepackage{tocloft}
\usepackage{appendix}

%------------------------------------------------------------------------------
% CONFIGURATION
%------------------------------------------------------------------------------
\geometry{left=2.5cm, right=2.5cm, top=3cm, bottom=3cm}

% En-têtes et pieds de page
\pagestyle{fancy}
\fancyhf{}
\fancyhead[LE,RO]{\thepage}
\fancyhead[RE]{\leftmark}
\fancyhead[LO]{\rightmark}
\renewcommand{\headrulewidth}{0.4pt}
\renewcommand{\footrulewidth}{0pt}

% Configuration des listings pour le code
\definecolor{codegreen}{rgb}{0,0.6,0}
\definecolor{codegray}{rgb}{0.5,0.5,0.5}
\definecolor{codepurple}{rgb}{0.58,0,0.82}
\definecolor{backcolour}{rgb}{0.97,0.97,0.97}

\lstdefinestyle{mystyle}{
    backgroundcolor=\color{backcolour},   
    commentstyle=\color{codegreen},
    keywordstyle=\color{blue}\bfseries,
    numberstyle=\tiny\color{codegray},
    stringstyle=\color{codepurple},
    basicstyle=\ttfamily\footnotesize,
    breakatwhitespace=false,         
    breaklines=true,                 
    captionpos=b,                    
    keepspaces=true,                 
    numbers=left,                    
    numbersep=5pt,                  
    showspaces=false,                
    showstringspaces=false,
    showtabs=false,                  
    tabsize=2,
    frame=single,
    rulecolor=\color{black}
}
\lstset{style=mystyle}

% Hyperliens
\hypersetup{
    colorlinks=true,
    linkcolor=blue!70!black,
    filecolor=magenta,      
    urlcolor=cyan,
    citecolor=green!60!black,
}

%------------------------------------------------------------------------------
% PAGE DE TITRE
%------------------------------------------------------------------------------
\begin{document}

\begin{titlepage}
    \centering
    
    % Logo de l'établissement (placeholder)
    % \includegraphics[width=0.3\textwidth]{logo_ecole.png}\\[1cm]
    \vspace{1cm}
    
    {\scshape\LARGE Nom de l'Établissement \par}
    \vspace{0.5cm}
    {\scshape\large Département Génie Informatique \par}
    \vspace{2cm}
    
    \rule{\textwidth}{1.5pt}\\[0.5cm]
    {\Huge\bfseries Application Web de Gestion Scolaire \par}
    \rule{\textwidth}{1.5pt}\\[1cm]
    
    {\Large\itshape Rapport de Projet \par}
    \vspace{1cm}
    
    {\large\textbf{Technologies :} Spring Boot 3 | Java 17 | Thymeleaf | Bootstrap 5 \par}
    \vspace{2cm}
    
    \begin{minipage}[t]{0.45\textwidth}
        \begin{flushleft}
            \textbf{Réalisé par :}\\
            Votre Nom\\
            Votre Numéro Étudiant
        \end{flushleft}
    \end{minipage}
    \hfill
    \begin{minipage}[t]{0.45\textwidth}
        \begin{flushright}
            \textbf{Encadré par :}\\
            Nom du Professeur\\
            Département d'Informatique
        \end{flushright}
    \end{minipage}
    
    \vfill
    
    {\large Année Académique 2024 -- 2025 \par}
    
\end{titlepage}

%------------------------------------------------------------------------------
% RÉSUMÉ / ABSTRACT
%------------------------------------------------------------------------------
\chapter*{Résumé}
\addcontentsline{toc}{chapter}{Résumé}

Ce rapport présente la conception et le développement d'une application web de gestion scolaire réalisée dans le cadre d'un projet académique. L'application, construite sur l'écosystème \textbf{Spring Boot 3} avec \textbf{Java 17}, implémente une architecture \textbf{MVC (Modèle-Vue-Contrôleur)} stricte et offre une gestion complète des entités principales d'un établissement scolaire : élèves, filières, cours et dossiers administratifs.

Les fonctionnalités clés incluent l'inscription automatisée des élèves avec génération dynamique de numéros d'identification, la gestion des relations complexes entre entités (OneToMany, ManyToMany, OneToOne), ainsi qu'une interface utilisateur moderne basée sur \textbf{Thymeleaf} et \textbf{Bootstrap 5}. La persistance des données est assurée par \textbf{Spring Data JPA} avec une base de données \textbf{H2} embarquée.

\textbf{Mots-clés :} Spring Boot, Java, MVC, JPA, Thymeleaf, Application Web, Gestion Scolaire

%------------------------------------------------------------------------------
% TABLE DES MATIÈRES
%------------------------------------------------------------------------------
\tableofcontents
\listoffigures
\listoftables
\lstlistoflistings
\newpage

%==============================================================================
% CHAPITRE 1 : INTRODUCTION
%==============================================================================
\chapter{Introduction}

\section{Contexte du Projet}
Dans le cadre de la digitalisation des processus administratifs des établissements scolaires, la nécessité de disposer d'outils informatiques performants et fiables devient primordiale. Ce projet s'inscrit dans cette optique en proposant une solution web moderne pour la gestion des inscriptions et du suivi académique des étudiants.

\section{Problématique}
La gestion manuelle ou semi-automatisée des dossiers étudiants présente plusieurs défis :
\begin{itemize}[noitemsep]
    \item Risque d'erreurs dans la saisie et le suivi des données
    \item Difficulté de traçabilité des inscriptions
    \item Absence de génération automatique d'identifiants uniques
    \item Complexité dans la gestion des relations (élèves-filières-cours)
\end{itemize}

\section{Objectifs}
Les objectifs de ce projet sont multiples :
\begin{enumerate}[noitemsep]
    \item Concevoir une architecture logicielle robuste et maintenable
    \item Implémenter les opérations CRUD pour toutes les entités métier
    \item Automatiser la création des dossiers administratifs
    \item Générer des identifiants d'inscription normalisés (format : \texttt{CODE\_FILIÈRE-ANNÉE-ID})
    \item Fournir une interface utilisateur intuitive et responsive
\end{enumerate}

\section{Organisation du Rapport}
Ce rapport s'articule autour des chapitres suivants :
\begin{itemize}[noitemsep]
    \item \textbf{Chapitre 2} : Analyse et spécifications des besoins
    \item \textbf{Chapitre 3} : Conception et architecture technique
    \item \textbf{Chapitre 4} : Réalisation et implémentation
    \item \textbf{Chapitre 5} : Tests et validation
    \item \textbf{Chapitre 6} : Conclusion et perspectives
\end{itemize}

%==============================================================================
% CHAPITRE 2 : ANALYSE ET SPÉCIFICATIONS
%==============================================================================
\chapter{Analyse et Spécifications}

\section{Analyse des Besoins Fonctionnels}

\subsection{Cas d'Utilisation Principaux}
Les principales fonctionnalités attendues sont présentées dans le Tableau~\ref{tab:cas_utilisation}.

\begin{table}[H]
    \centering
    \caption{Cas d'utilisation principaux}
    \label{tab:cas_utilisation}
    \begin{tabularx}{\textwidth}{|l|X|}
        \hline
        \textbf{Module} & \textbf{Fonctionnalités} \\
        \hline
        Gestion des Filières & Créer, Lire, Modifier, Supprimer une filière \\
        \hline
        Gestion des Cours & CRUD des cours, Association à une filière \\
        \hline
        Gestion des Élèves & CRUD, Affectation filière, Inscription aux cours, Génération automatique du dossier \\
        \hline
        Dossier Administratif & Création automatique, Génération du numéro d'inscription \\
        \hline
        Navigation & Page d'accueil, Menu de navigation, Liens contextuels \\
        \hline
    \end{tabularx}
\end{table}

\subsection{Règles de Gestion}
\begin{enumerate}[label=\textbf{RG\arabic*}:, noitemsep]
    \item Un élève appartient à une et une seule filière
    \item Un élève peut suivre plusieurs cours de sa filière
    \item Chaque élève possède un dossier administratif unique (relation 1:1)
    \item Le numéro d'inscription est généré automatiquement selon le format : \\ \texttt{<CODE\_FILIÈRE>-<ANNÉE>-<ID\_ÉLÈVE>}
    \item La suppression d'un élève entraîne la suppression de son dossier (Cascade)
\end{enumerate}

\section{Besoins Non Fonctionnels}

\begin{table}[H]
    \centering
    \caption{Exigences non fonctionnelles}
    \label{tab:non_fonctionnel}
    \begin{tabularx}{\textwidth}{|l|X|}
        \hline
        \textbf{Critère} & \textbf{Description} \\
        \hline
        Performance & Temps de réponse inférieur à 2 secondes \\
        \hline
        Maintenabilité & Code structuré en couches (Controller, Service, Repository) \\
        \hline
        Portabilité & Application exécutable sur tout système disposant d'une JVM 17+ \\
        \hline
        Ergonomie & Interface responsive adaptée aux différents écrans \\
        \hline
    \end{tabularx}
\end{table}

%==============================================================================
% CHAPITRE 3 : CONCEPTION
%==============================================================================
\chapter{Conception et Architecture}

\section{Choix Technologiques}

\subsection{Stack Technique}
Le Tableau~\ref{tab:stack} présente l'ensemble des technologies utilisées.

\begin{table}[H]
    \centering
    \caption{Stack technologique du projet}
    \label{tab:stack}
    \begin{tabular}{|l|l|l|}
        \hline
        \textbf{Catégorie} & \textbf{Technologie} & \textbf{Version} \\
        \hline
        Langage & Java & 17 (LTS) \\
        Framework Backend & Spring Boot & 3.2.1 \\
        Build Tool & Maven & 3.9.x \\
        ORM & Hibernate (via Spring Data JPA) & 6.x \\
        Base de données & H2 Database & 2.x \\
        Template Engine & Thymeleaf & 3.1.x \\
        Framework CSS & Bootstrap & 5.3 \\
        Utilitaire & Lombok & 1.18.30 \\
        \hline
    \end{tabular}
\end{table}

\subsection{Justification des Choix}
\begin{itemize}
    \item \textbf{Spring Boot 3} : Framework mature offrant une configuration minimale, une injection de dépendances robuste et un écosystème riche.
    \item \textbf{Java 17} : Version LTS offrant performances et fonctionnalités modernes (Records, Pattern Matching).
    \item \textbf{H2 Database} : Base de données embarquée facilitant le développement et les tests sans configuration externe.
    \item \textbf{Thymeleaf} : Moteur de template intégré nativement à Spring, permettant un rendu côté serveur sécurisé.
    \item \textbf{Lombok} : Réduction significative du code boilerplate (getters, setters, constructeurs).
\end{itemize}

\section{Architecture Logicielle}

\subsection{Architecture en Couches (Layered Architecture)}
L'application suit le pattern MVC enrichi d'une couche Service :

\begin{figure}[H]
    \centering
    % \includegraphics[width=0.7\textwidth]{architecture_couches.png}
    \fbox{\begin{minipage}{0.8\textwidth}
        \centering
        \vspace{1.5cm}
        \textbf{[Insérer Diagramme d'Architecture en Couches]}\\[0.5cm]
        \texttt{Controller $\rightarrow$ Service $\rightarrow$ Repository $\rightarrow$ Database}
        \vspace{1.5cm}
    \end{minipage}}
    \caption{Architecture en couches de l'application}
    \label{fig:architecture}
\end{figure}

\begin{itemize}
    \item \textbf{Controller} : Réception des requêtes HTTP, préparation du Model, sélection de la Vue
    \item \textbf{Service} : Logique métier, orchestration des opérations, transactions
    \item \textbf{Repository} : Accès aux données via Spring Data JPA
    \item \textbf{Entity} : Mapping objet-relationnel des tables
\end{itemize}

\subsection{Structure des Packages}
\begin{lstlisting}[language=bash, caption={Organisation des packages}]
com.example.school
├── controller
│   ├── HomeController.java
│   ├── EleveController.java
│   ├── FiliereController.java
│   └── CoursController.java
├── service
│   ├── EleveService.java
│   ├── FiliereService.java
│   └── CoursService.java
├── repository
│   ├── EleveRepository.java
│   ├── FiliereRepository.java
│   ├── CoursRepository.java
│   └── DossierAdministratifRepository.java
├── entity
│   ├── Eleve.java
│   ├── Filiere.java
│   ├── Cours.java
│   └── DossierAdministratif.java
├── DataSeeder.java
└── SchoolApplication.java
\end{lstlisting}

\section{Modèle de Données}

\subsection{Diagramme de Classes UML}

\begin{figure}[H]
    \centering
    % \includegraphics[width=0.9\textwidth]{diagramme_classes.png}
    \fbox{\begin{minipage}{0.9\textwidth}
        \centering
        \vspace{3cm}
        \textbf{[Insérer Diagramme de Classes UML]}\\[0.5cm]
        Relations : Filiere (1) $\leftrightarrow$ (*) Eleve\\
        Filiere (1) $\leftrightarrow$ (*) Cours\\
        Eleve (*) $\leftrightarrow$ (*) Cours\\
        Eleve (1) $\leftrightarrow$ (1) DossierAdministratif
        \vspace{3cm}
    \end{minipage}}
    \caption{Diagramme de classes du domaine métier}
    \label{fig:diag_classes}
\end{figure}

\subsection{Dictionnaire de Données}

\begin{table}[H]
    \centering
    \caption{Entité Eleve}
    \begin{tabular}{|l|l|l|l|}
        \hline
        \textbf{Attribut} & \textbf{Type} & \textbf{Contrainte} & \textbf{Description} \\
        \hline
        id & Long & PK, Auto & Identifiant unique \\
        nom & String & Not Null & Nom de famille \\
        prenom & String & Not Null & Prénom \\
        filiere & Filiere & FK, ManyToOne & Filière d'appartenance \\
        dossierAdministratif & DossierAdministratif & OneToOne, Cascade & Dossier associé \\
        cours & List<Cours> & ManyToMany & Cours suivis \\
        \hline
    \end{tabular}
\end{table}

\begin{table}[H]
    \centering
    \caption{Entité DossierAdministratif}
    \begin{tabular}{|l|l|l|l|}
        \hline
        \textbf{Attribut} & \textbf{Type} & \textbf{Contrainte} & \textbf{Description} \\
        \hline
        id & Long & PK, Auto & Identifiant unique \\
        numeroInscription & String & Unique & Format: CODE-ANNÉE-ID \\
        dateCreation & LocalDate & Not Null & Date de création \\
        eleve & Eleve & OneToOne & Élève propriétaire \\
        \hline
    \end{tabular}
\end{table}

%==============================================================================
% CHAPITRE 4 : RÉALISATION
%==============================================================================
\chapter{Réalisation et Implémentation}

\section{Configuration du Projet}

\subsection{Fichier pom.xml (Dépendances Maven)}
\begin{lstlisting}[language=XML, caption={Extrait des dépendances Maven}]
<parent>
    <groupId>org.springframework.boot</groupId>
    <artifactId>spring-boot-starter-parent</artifactId>
    <version>3.2.1</version>
</parent>

<dependencies>
    <dependency>
        <groupId>org.springframework.boot</groupId>
        <artifactId>spring-boot-starter-web</artifactId>
    </dependency>
    <dependency>
        <groupId>org.springframework.boot</groupId>
        <artifactId>spring-boot-starter-data-jpa</artifactId>
    </dependency>
    <dependency>
        <groupId>org.springframework.boot</groupId>
        <artifactId>spring-boot-starter-thymeleaf</artifactId>
    </dependency>
    <dependency>
        <groupId>com.h2database</groupId>
        <artifactId>h2</artifactId>
        <scope>runtime</scope>
    </dependency>
    <dependency>
        <groupId>org.projectlombok</groupId>
        <artifactId>lombok</artifactId>
        <optional>true</optional>
    </dependency>
</dependencies>
\end{lstlisting}

\subsection{Configuration application.properties}
\begin{lstlisting}[caption={Configuration de l'application}]
spring.application.name=SchoolManagement
server.port=8081

# H2 Database
spring.datasource.url=jdbc:h2:mem:school_db
spring.datasource.driverClassName=org.h2.Driver
spring.datasource.username=sa
spring.datasource.password=

# JPA/Hibernate
spring.jpa.database-platform=org.hibernate.dialect.H2Dialect
spring.jpa.hibernate.ddl-auto=update

# H2 Console
spring.h2.console.enabled=true
spring.h2.console.path=/h2-console
\end{lstlisting}

\section{Implémentation des Entités JPA}

\begin{lstlisting}[language=Java, caption={Entité Eleve avec annotations JPA}]
@Entity
@Data
@NoArgsConstructor
@AllArgsConstructor
public class Eleve {
    @Id
    @GeneratedValue(strategy = GenerationType.IDENTITY)
    private Long id;

    private String nom;
    private String prenom;

    @ManyToOne
    @JoinColumn(name = "filiere_id")
    private Filiere filiere;

    @ManyToMany
    @JoinTable(
        name = "eleve_cours",
        joinColumns = @JoinColumn(name = "eleve_id"),
        inverseJoinColumns = @JoinColumn(name = "cours_id")
    )
    @ToString.Exclude
    private List<Cours> cours = new ArrayList<>();

    @OneToOne(cascade = CascadeType.ALL)
    @JoinColumn(name = "dossier_administratif_id")
    private DossierAdministratif dossierAdministratif;
}
\end{lstlisting}

\section{Logique Métier : Service d'Inscription}

\begin{lstlisting}[language=Java, caption={Méthode inscrireEleve - Cœur de la logique métier}]
@Transactional
public void inscrireEleve(Eleve eleve, Long filiereId) {
    // 1. Récupération de la filière
    Filiere filiere = filiereRepository.findById(filiereId)
            .orElseThrow(() -> new RuntimeException("Filière introuvable"));

    // 2. Association élève-filière
    eleve.setFiliere(filiere);

    // 3. Création du dossier administratif
    DossierAdministratif dossier = new DossierAdministratif();
    dossier.setDateCreation(LocalDate.now());
    dossier.setEleve(eleve);
    eleve.setDossierAdministratif(dossier);

    // 4. Première sauvegarde pour générer l'ID
    eleve = eleveRepository.saveAndFlush(eleve);

    // 5. Génération du numéro d'inscription
    int annee = LocalDate.now().getYear();
    String numeroInscription = String.format("%s-%d-%d", 
            filiere.getCode(), annee, eleve.getId());

    // 6. Mise à jour et sauvegarde finale
    eleve.getDossierAdministratif().setNumeroInscription(numeroInscription);
    eleveRepository.save(eleve);
}
\end{lstlisting}

\section{Interfaces Utilisateur}

\subsection{Page d'Accueil}
\begin{figure}[H]
    \centering
    % \includegraphics[width=0.9\textwidth]{screenshots/accueil.png}
    \fbox{\begin{minipage}{0.9\textwidth}
        \centering
        \vspace{3cm}
        \textbf{[Capture d'écran : Page d'accueil avec tableau de bord]}
        \vspace{3cm}
    \end{minipage}}
    \caption{Interface de la page d'accueil}
    \label{fig:accueil}
\end{figure}

\subsection{Liste des Élèves}
\begin{figure}[H]
    \centering
    % \includegraphics[width=0.9\textwidth]{screenshots/liste_eleves.png}
    \fbox{\begin{minipage}{0.9\textwidth}
        \centering
        \vspace{3cm}
        \textbf{[Capture d'écran : Liste des élèves avec actions CRUD]}
        \vspace{3cm}
    \end{minipage}}
    \caption{Interface de gestion des élèves}
    \label{fig:liste_eleves}
\end{figure}

\subsection{Détails d'un Élève et Dossier Administratif}
\begin{figure}[H]
    \centering
    % \includegraphics[width=0.9\textwidth]{screenshots/details_eleve.png}
    \fbox{\begin{minipage}{0.9\textwidth}
        \centering
        \vspace{3cm}
        \textbf{[Capture d'écran : Vue détaillée avec dossier et cours]}
        \vspace{3cm}
    \end{minipage}}
    \caption{Page de détails d'un élève}
    \label{fig:details_eleve}
\end{figure}

\subsection{Gestion des Filières et Cours}
\begin{figure}[H]
    \centering
    \begin{subfigure}[b]{0.48\textwidth}
        % \includegraphics[width=\textwidth]{screenshots/filieres.png}
        \fbox{\begin{minipage}{\textwidth}
            \centering
            \vspace{2cm}
            \textbf{[Liste Filières]}
            \vspace{2cm}
        \end{minipage}}
        \caption{Liste des filières}
    \end{subfigure}
    \hfill
    \begin{subfigure}[b]{0.48\textwidth}
        % \includegraphics[width=\textwidth]{screenshots/cours.png}
        \fbox{\begin{minipage}{\textwidth}
            \centering
            \vspace{2cm}
            \textbf{[Liste Cours]}
            \vspace{2cm}
        \end{minipage}}
        \caption{Liste des cours}
    \end{subfigure}
    \caption{Interfaces de gestion des filières et cours}
    \label{fig:filieres_cours}
\end{figure}

%==============================================================================
% CHAPITRE 5 : TESTS ET VALIDATION
%==============================================================================
\chapter{Tests et Validation}

\section{Tests Fonctionnels}

\subsection{Scénarios de Test}

\begin{table}[H]
    \centering
    \caption{Matrice des tests fonctionnels}
    \label{tab:tests}
    \begin{tabularx}{\textwidth}{|l|X|c|}
        \hline
        \textbf{Test} & \textbf{Description} & \textbf{Résultat} \\
        \hline
        TF01 & Création d'une nouvelle filière & \checkmark \\
        TF02 & Ajout d'un cours associé à une filière & \checkmark \\
        TF03 & Inscription d'un élève avec génération auto du dossier & \checkmark \\
        TF04 & Vérification du format du numéro d'inscription & \checkmark \\
        TF05 & Ajout d'un cours à un élève existant & \checkmark \\
        TF06 & Suppression d'un élève (cascade sur dossier) & \checkmark \\
        TF07 & Navigation entre toutes les pages & \checkmark \\
        \hline
    \end{tabularx}
\end{table}

\section{Compilation et Déploiement}

\subsection{Commandes d'Exécution}
\begin{lstlisting}[language=bash, caption={Compilation et lancement}]
# Compilation du projet
./mvnw clean package -DskipTests

# Exécution de l'application
./mvnw spring-boot:run

# Accès à l'application
# URL: http://localhost:8081

# Console H2 (debug)
# URL: http://localhost:8081/h2-console
# JDBC URL: jdbc:h2:mem:school_db
# User: sa | Password: (vide)
\end{lstlisting}

%==============================================================================
% CHAPITRE 6 : CONCLUSION
%==============================================================================
\chapter{Conclusion et Perspectives}

\section{Bilan du Projet}
Ce projet a permis de mettre en pratique les concepts fondamentaux du développement web Java enterprise :
\begin{itemize}
    \item Maîtrise de l'architecture MVC avec Spring Boot
    \item Implémentation des relations JPA complexes (1:1, 1:N, N:N)
    \item Développement d'interfaces utilisateur modernes avec Thymeleaf et Bootstrap
    \item Gestion transactionnelle et logique métier dans la couche Service
\end{itemize}

\section{Difficultés Rencontrées}
\begin{itemize}
    \item Configuration initiale de Lombok avec le compilateur Maven
    \item Gestion des relations bidirectionnelles et prévention des boucles infinies
    \item Mise en place du mapping correct pour les relations ManyToMany
\end{itemize}

\section{Améliorations Futures}
\begin{enumerate}
    \item \textbf{Sécurité} : Intégration de Spring Security avec authentification JWT
    \item \textbf{Base de données} : Migration vers MySQL/PostgreSQL en production
    \item \textbf{API REST} : Exposition des services via une API RESTful
    \item \textbf{Tests} : Ajout de tests unitaires (JUnit) et d'intégration
    \item \textbf{Conteneurisation} : Dockerisation de l'application
    \item \textbf{CI/CD} : Mise en place d'un pipeline d'intégration continue
\end{enumerate}

%==============================================================================
% ANNEXES
%==============================================================================
\appendix
\chapter{Guide d'Installation}

\section{Prérequis}
\begin{itemize}
    \item JDK 17 ou supérieur
    \item Maven 3.6+ (ou utiliser le wrapper Maven inclus)
    \item IDE recommandé : IntelliJ IDEA, Eclipse ou VS Code
\end{itemize}

\section{Installation}
\begin{enumerate}
    \item Cloner ou extraire le projet
    \item Ouvrir un terminal dans le dossier du projet
    \item Exécuter : \texttt{./mvnw spring-boot:run}
    \item Accéder à : \url{http://localhost:8081}
\end{enumerate}

\chapter{Références}
\begin{itemize}
    \item Documentation officielle Spring Boot : \url{https://spring.io/projects/spring-boot}
    \item Guide Spring Data JPA : \url{https://spring.io/projects/spring-data-jpa}
    \item Documentation Thymeleaf : \url{https://www.thymeleaf.org/documentation.html}
    \item Bootstrap 5 : \url{https://getbootstrap.com/docs/5.3/}
\end{itemize}

\end{document}
